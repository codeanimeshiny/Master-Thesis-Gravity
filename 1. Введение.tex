\section{Introduction} 
\label{sec: intro}


Appearance of Gravity model had significantly complemented range of theories describing motives of international trade. Along with that, due to intuitive clarity, simplicity and decent practical application gravity model is now hold as one of the main tools in many research fields. By its flexibility, gravity model is widely applied not only in estimation of international trade, but in social studies \parencite{beine2016practitioners}, transportation and logistics related research \parencite{host2019trade, mendes2020impact}, as well as in other directions. 

Currently, the patterns of trade across countries are described exhaustively within the framework of the traditional gravity model. For example, gravity best practice is thoroughly reviewed and summarized by \textcite{yotov2016advanced}. At the same time, the hypothesis arises: whether gravity model can be as good for studying trade dependencies in smaller territorial units, for example, cities or metropolitan areas? There is no extensive amount of unambiguous and consistent research on this question, so this serves as the basis for our exploration. Moreover, particularly in international trade vast range of gravity applications are primarily relying on measurable and direct economic indicators such as GDP and trade volumes. However, at the smaller than country-level scale, such data is often unavailable or incomplete. For now unclear if it is possible to replace key control variables in gravity model with indirect indicators (proxies)? And if so, how successfully can these proxies replace traditional attributes?  We directly relate these ideas to the subject matters of our research, focused on studying the performance of the gravity model on a smaller areas.

Projecting the international trade gravity model from the national level to a smaller scale presents challenges. The complexity lies in the fact that it is unclear how to identify trading partners for such a study configuration. Potentially, regions or cities from neighboring countries might be considered as minor trading partners. However, such scenarios are often too heterogeneous and unlikely to fit within a single framework, except when analyzing specific instances of cross-border trade \parencite{capello2018measuring}. Thus, in our research the focus shifts from international to domestic trade. This shift introduces another challenge — defining what constitutes trade within a country's internal regions. In a broad sense, goods turnover between territories, expressed in monetary terms, can be considered as a representation of trade flow. In our study, we do not delve into the specific methods for calculating the monetary value of trade turnover between territories, as this falls beyond the scope of our research. However, by utilizing real data on freight flows between U.S. economic zones, we adopt the concept of commodity freight flow (measured in dollars as sum of its goods values). Magnitude of such freight flows captures the scale of the overall "trade flow" between territories \parencite{coughlin2013international, strocko2014freight}. With that our core research question might be stated as follows: Can a gravity model successfully describe freight flows data between major economic zones of a United States?

As mentioned earlier, to achieve our research objectives, we also seek to validate the feasibility of addressing data insufficiency by incorporating proxy variables into the gravity model. We attempt to estimate and analyze freight flows flows using proxies derived from satellite imagery. Namely, we propose two key proxies: relative freight traffic density and nighttime light intensity. Recent advances in object detection by machine learning and remote sensing have demonstrated significant promise for analyzing urban traffic patterns \parencite{leitloff2006automatic, zambanini2020detection}. Аpplication of vehicle detection algorithms can be adapted to targeted freight traffic analysis. Therefore, freight traffic density might be derived from satellite imagery using pre-trained machine learning model and serve as a measure of either "trade" volume itself or economic activity of territory which in turn is presented by GDP in original model. Nighttime light intensity has been repeatedly used in economic research as an indirect indicator of a region's economic activity and development as well \parencite{sutton2007estimation, henderson2012measuring, mellander2015night}. 

Hence, the objectives of this research are twofold. First is to validate original gravity model on US inter-regional data. Second is to implement gravity model with trade-related proxy data derived from satellite images by means of innovative techniques such as object detection algorithms. At last we evaluate the relevance of such approach and discuss its limitations. Overall, this study contributes to the field of spatial economics by offering a scalable and experimental methodology for analyzing trade flows in data-scarce environments.

The following parts of the paper are structured as follows. The \hyperref[sec: litrev]{\emph{Literature Review}} section provides a brief overview of Gravity model and its implementation principles. \hyperref[sec: litrev]{\emph{Literature Review}} section also covers results and features of few works closely related to the topic of research. Observation of several papers devoted to proxy measures in economic studies is given in the literature part as well. The \hyperref[sec: data]{\emph{Data}} section is a quantitative and qualitative description of data utilized in practical part of our work. This chapter also provides an insight about data that might be be useful for conducting a related study. In the \hyperref[sec: method]{\emph{Methodology}} section we talk about the modeling framework that we adhere to, as well as the evaluation procedure and the specifications of the models we use to obtain our results. In \hyperref[sec: res]{\emph{Results}} part we interpret results obtained after running the constructed model specifications, while the \hyperref[sec: disc]{\emph{Discussion}} section highlights limitations of executed approach and ideas for expanding research. Finally, \hyperref[sec:conclusions]{\emph{Conclusion}} summarizes main ideas and key results of the study.
