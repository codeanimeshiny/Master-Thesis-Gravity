\section{Literature Review}
\label{sec: litrev}

\subsection{Gravity Model: Common Practice}

The modern understanding of the gravity model was significantly shaped by \textcite{bergstrand1985gravity} and later refined by \textcite{anderson2003gravity}. The latter's work was pivotal in establishing the role of multilateral resistance, emphasizing that trade flows depend not only on bilateral trade costs but also on a partner’s overall trade environment. \textcite{anderson2003gravity} introduced the Structural Gravity equation, which explicitly incorporates multilateral resistance through multilateral price indices.

To effectively control for multilateral resistance, \textcite{anderson2003gravity} proposed the use of exporter and importer fixed effects interacted with time, which capture both observable and unobservable trade barriers. This approach allows for a more precise estimation of trade flows by filtering out exporter-specific and importer-specific influences that affect all trading partners. Additionally, \textcite{anderson2011gravity} demonstrated that enriching the gravity model with trade friction indicators — such as historical ties (e.g., colonial relationships) and shared linguistic or cultural factors — enhances its explanatory power. It is worth mention that in the case of domestic trade relationships, a whole set of such control variables may be clearly impractical. Therefore, we pay close attention to the model specifications being constructed and carefully consider this limitation when interpreting the results. Further, \textcite{yotov2016advanced} argued that incorporating pair fixed effects improves model performance by controlling for persistent, bilateral trade costs. Unlike traditional bilateral trade variables, pair fixed effects provide a more reliable measure of trade barriers by isolating the unique relationship between two trade partners over time. This refinement strengthens the empirical robustness of the gravity model, making it a cornerstone of modern trade analysis.

Another core pillar of gravity is the estimation method. Most studies comparing log-linear models with GLM confirm several key issues. First of all, the assumption of homoscedasticity, rarely valid for spatial data, biases standard errors and distorts coefficient significance in OLS estimations \parencite{egger2015multi}. Secondly, log-linear estimation omits zero trade flows, preventing analysis of factors driving the absence of trade and often leading to overestimated trade potential \parencite{martinez2013log, egger2016glm}. Finally, OLS provides an average effect of model factors on trade volume, making it unsuitable for analyzing trade potential across different parts of the distribution, where a non-linear approach is needed \parencite{silva2006log}. In view of this, researchers highlight that OLS-based models often produce biased estimates and are significantly less accurate than GLM alternatives. To address these limitations, the literature suggests alternative methods, including the Tobit model, Gamma pseudo-maximum likelihood (GPML), and Non-linear least squares (NLS). However, each of these approaches has drawbacks, which the Poisson pseudo-maximum likelihood (PPML) model effectively mitigates.

In their seminal work, \textcite{silva2006log} demonstrated the superiority of Poisson Pseudo-Maximum Likelihood (PPML) estimation. Authors had shown its resilience against heteroskedasticity and its suitability for the multiplicative form of the gravity model. Unlike log-linearized models, which lose information on zero trade flows, PPML preserves this data. Moreover, \textcite{silva2006log} found that PPML yields significantly different coefficient estimates compared to OLS and alternative methods (e.g., Tobit, NLS) when applied to a Van Wincoop-style gravity model \parencite{anderson2003gravity}, which correctly accounts for multilateral resistance. To validate the robustness of their results, \textcite{silva2006log} conducted a RESET test resistant to heteroskedasticity, confirming PPML’s advantages. Further confirmation of PPML’s robustness came by \parencite{silva2011further}. Key findings of Silva and Tenreyro were validated by many papers, for instance, \textcite{motta2019estimating}, who compared different estimation methods in a non-trade context — health insurance data — using several approaches. Their findings aligned with those of \textcite{silva2006log}, revealing significant coefficient differences between PPML and other estimators, reinforcing the idea that heteroskedasticity is a key driver of these discrepancies. Eventually, PPML has firmly established itself as a leading method in the empirical analysis of gravity models.

Overall, key suggestions that discovered in the relevant literature on the subject of gravity are outlined in the practical section of our work.

\subsection{Gravity Model: Domestic Trade Investigation}

%For some time gravity models have been estimated using only international trade flow data. However, recent studies emphasize the importance of incorporating intra-national (i.e., domestic) trade flows \parencite{campos2021structural}. 

Yotov, as a maitre of gravity studies, notes that most empirical gravity model applications overlook domestic trade flows and exclude them from estimation samples \parencite{yotov2022role}. Although domestic trade flows might be constructed by adopting various unpretentious calculations \parencite{campos2021structural}, the lack of direct and ready-to-use data is a common obstacle on the way of domestic trade gravity modeling \parencite{yotov2022role}.

Nevertheless, \textcite{yotov2022role} provides a well-argued discussion of how domestic trade flows are a fundamental component of theoretical (and therefore practical) gravity framework. For instance, incorporating domestic trade flows enables the estimation of heterogeneous trade costs at local level. In \textcite{agnosteva2019intra} observation of Canada’s provinces shows that variations in relative border frictions impact regions differently. Remote provinces experiencing greater trade barriers, while large centered provinces benefit from more favorable trade conditions. 

Moreover, intra-national trade is beneficial in assessing the role of borders and the "home bias" in trade. And gravity framework was applied in a series of investigations on the topic of co-called "border puzzle" that was in the center of the scope (see \cite{mccallum1995national}; \cite{wolf2000intranational}). The "puzzle" arises because the estimated border effects — often indicating that trade within a country is many times higher than across borders — seem too large to be explained by tariffs, policy barriers, or observable frictions alone. Later research (e.g., \cite{anderson2003gravity}) addressed the puzzle by incorporating multilateral resistance terms. However, the border puzzle remains an active theme in trade research. What is more interesting for us, although initially it was about international trade flows, the researchers transferred the issue of "border puzzle" and "home bias" to the local, intra-national context. Several studies (for example, \cite{nitsch2000national}; \cite{chen2004intra}) treat an issue regarding the European Union, offering broader valuable insights into the geographical patterns of border-effects. We, however, place more emphasis on research focused on the United States \parencite{hillberry2003intranational, millimet2007state, coughlin2013international, crafts2014geography}. These papers mostly examine why trade within U.S. states is significantly greater than trade between states, despite the absence of major trade barriers.

Frankly speaking, the great value for us is not so much the in-depth problematics of these research (although of course partially it is), but rather the methodology and data that can be used for US interregional gravity construction. \textcite{hillberry2003intranational} applied a fairly simple methodology by building an OLS regression on the value of commodity shipment flows. Among the control variables were origin and destination state outputs, but substituting it by fixed effects of the "exporting" and "importing" territory reduced unexplained variation in the model. Dummy of in-state flow, and distance measured in various ways were also incorporated. Work done by \textcite{millimet2007state} is much broader in terms of applied methodology compared to former paper. In addition to the individual and, remarkably, pair fixed effects, which are for now already part of the traditional controls in gravity, authors have significantly broaden empirical part. Particularly by incorporating past trade levels in a dynamic panel model and accounting for internal migration flows. \textcite{millimet2007state} also compared OLS with GMM estimator, which handles the presence of heteroscedasticity. Their findings suggest that historical trade patterns and migration-driven network effects might play a significant role in shaping trade flows between states. The same "home bias" issue was addressed in later \textcite{coughlin2013international}, still the approach to modeling remained roughly the same, but we outline the significance of dummy responsible for the common border of trading territories. \textcite{coughlin2013international} estimate the model using OLS with individual fixed and, notably, random effects which probably makes sense: in some specifications, random effects are used to address collinearity issues. \textcite{crafts2014geography} examine how home bias in U.S. domestic trade has changed over time by comparing data from 1949 and 2007. Eventually, this paper utilizing Poisson Pseudo-Maximum Likelihood (PPML), recommended by \textcite{silva2006log}. Among other classical gravity covariates, similarly to the work of \textcite{millimet2007state}, the measure of remoteness of the territory is implemented in the model as a control variable (an indicator weighted by the distance of region \textit{i} from other regions and the total output of the region \textit{i}). However, the model specifications lack pairwise fixed effects, although individual ones are included.

Ultimately, these studies provide a more nuanced understanding of geographic frictions in intra-national trade and challenge the magnitude of estimates of home bias. Yet, the methodology of these studies is imperfect, as our brief methodological review suggests that the models exhibit misspecification — either in terms of the included effects or the estimation method used. In turn, in our research we aim to focus on addressing the methodological shortcomings that we have managed to detect in the reviewed readings. Specifically, we aim to perform a rigorous intra-national gravity analysis using the most robust and effective estimation method (PPML) while constructing model specifications that incorporate all relevant control variables and effects.

Something that unites aforementioned works, in addition to a more or less similar modelling approach, is data. Researchers \parencite{hillberry2003intranational, millimet2007state, coughlin2013international, crafts2014geography} use data from U.S. Commodity Flow Survey (CFS)\footnote{ \href{https://www.bts.gov/cfs}{Commodity Flow Survey}} for several periods of time. Conducted every five years by the U.S. Census Bureau, the CFS collects data from a sample of shipments originating from U.S. mining, manufacturing, and wholesale establishments. The reported shipment characteristics include: weight, value, commodity classification, along with the origin and destination points codes and the actual shipping distance between them. We find the presence of this data in empirical works encouraging for us for two reasons. First, the use of these data actually confirms the validity of our hypothesis about the possibility of presenting data on trade in the interior of the country (in particular, the United States) through a picture of commodity freight flows. Secondly, the very existence of such data greatly facilitates the practical task, eliminating the need to reproduce trade data manually, through calculations. We talk about the data utilized in our framework in more details in the next chapter.

\subsection{Satellite Data: Proxy in Economic Modelling}

The use of nighttime light (NTL) data as a proxy for economic activity, particularly Gross Domestic Product (GDP), has gained significant attention in recent years. Its potential to provide timely and spatially detailed insights, especially in regions where traditional economic data are scarce or unreliable, is justified. Studies have demonstrated that NTL data, derived from satellites such as the Defense Meteorological Satellite Program (DMSP) and the Visible Infrared Imaging Radiometer Suite (VIIRS), exhibit strong correlations with GDP at various spatial scales, from national to sub-national levels \parencite{henderson2012measuring, addison2015nighttime, bickenbach2016night, gibson2021nighttime}. For instance, \textcite{henderson2012measuring} argue that changes in NTL intensity can serve as an alternative measure of economic growth, particularly at subnational levels. Similarly, \textcite{sutton2007estimation} demonstrate that NTL can provide reliable estimates of GDP at both national and regional levels, though the accuracy varies depending on urbanization and economic structure, suggesting that NTL is more effective in urbanized areas with concentrated economic activity. This fits well with our idea of considering urbanized active economic zones in the United States, which are territories smaller than an entire state (usually one or more large urban agglomerations). At the same time, however, several invesigations show the relationship between NTL and GDP is less stable when examining temporal changes within regions, indicating that NTL data may be more suitable for capturing spatial differences in economic activity rather than short-term fluctuations \parencite{bickenbach2016night, gibson2021nighttime}. Work of \textcite{mellander2015night} further elaborates on this by finding that NTL correlates strongly with population density and urbanization. \textcite{mellander2015night}, however, underlines NTL's relationship with overall economic output is weaker, suggesting that NTL functions better as a measure of economic concentration rather than direct productivity. Either way, based on the evidence that this approach has already gained a strong foothold in the field of economic modeling, we will incorporate this proxy-data into our model to address the question of variable substitutability in the gravity model. Recent advancements in NTL data, such as the VIIRS Version 2 (V2 VNL) products, have improved the accuracy of NTL by reducing noise and providing more consistent measurements over time \parencite{gibson2021nighttime}.  In the context of our gravity model of freight flows these improvements are particularly relevant for applications at finer spatial scales. However, it is important to acknowledge the limitations of NTL data, particularly its potential underestimation of economic activity in less densely populated or agricultural regions. Overall, while NTL data offer a promising alternative to traditional GDP measures, their application should be carefully considered in light of these limitations, particularly when used to model economic interactions in metropolitan areas within a gravity framework.

%Methodological refinements, such as using radiance-calibrated NTL data, have been shown to improve its accuracy as a proxy for economic activity, making it a more robust tool for economic analysis \parencite{addison2015nighttime, mellander2015night}. Overall, while NTL data offer a promising alternative to traditional GDP measures, their application should be carefully considered in light of these limitations, particularly when used to model economic interactions in metropolitan areas within a gravity framework.

Alongside this, novel advances in object detection by machine learning and remote sensing have demonstrated significant promise for analyzing urban and intercity traffic patterns. Convolutional neural networks (CNNs) and other deep learning methods are being employed to identify and track vehicles from aerial imagery and satellite data \parencite{leitloff2006automatic, zambanini2020detection}. Аpplication of these car detection algorithms can be adapted to targeted freight traffic analysis. By isolating and monitoring freight vehicles from general cargo-traffic using Sentinel-2 data and advanced image recognition algorithms, it becomes feasible to estimate freight flows \parencite{fisser2022detecting}. Our research entails that freight movement serves as a proxy for economic activity, and its automated detection could provide critical insights into regional trade patterns and infrastructure use.
